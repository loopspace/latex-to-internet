% Markdown, prefix: mkdn

% We do a lot of redefining

\msg_redirect_module:nnn { LaTeX / xparse } { info }    { none }
\msg_redirect_module:nnn { LaTeX / xparse } { warning } { none }

% Sets the section levels (and other headings)
% Global offset:
\int_new:N \l__mkdn_secoff_int
\int_set:Nn \l__mkdn_secoff_int {1}
\int_new:N \l__mkdn_chapter_int
\int_new:N \l__mkdn_section_int
\int_new:N \l__mkdn_subsection_int
\int_new:N \l__mkdn_subsubsection_int
\int_new:N \l__mkdn_paragraph_int
\int_new:N \l__mkdn_subparagraph_int
\int_set:Nn \l__mkdn_chapter_int {0}
\int_set:Nn \l__mkdn_section_int {1}
\int_set:Nn \l__mkdn_subsection_int {2}
\int_set:Nn \l__mkdn_subsubsection_int {3}
\int_set:Nn \l__mkdn_paragraph_int {4}
\int_set:Nn \l__mkdn_subparagraph_int {5}

% Define the options for the markdown module
\keys_define:nn {markdown options}
{
  section~ offset .int_set:N = \l__mkdn_secoff_int,
  chapter~ level .int_set:N = \l__mkdn_chapter_int,
  section~ level .int_set:N = \l__mkdn_section_int,
  subsection~ level .int_set:N = \l__mkdn_subsection_int,
  subsubsection~ level .int_set:N = \l__mkdn_subsubsection_int,
  paragraph~ level .int_set:N = \l__mkdn_paragraph_int,
  subparagraph~ level .int_set:N = \l__mkdn_subparagraph_int
}

% Get and set the options for the markdown module
\prop_get:NnNT  \l__txt_popts_prop {markdown} \l_tmpa_tl
{
  \keys_set:nV {markdown options} \l_tmpa_tl
}

\RenewDocumentCommand \emph {m} {*#1*}
\RenewDocumentCommand \textit {m} {*#1*}
\RenewDocumentCommand \textbf {m} {**#1**}


% If the class defines a `\chapter` command, we need to do so also
% Otherwise, all the section commands step up one on the level
\tl_if_eq:VnTF \l__txt_class_tl {report}
{
  \NewDocumentCommand \chapter {s m}
  {
    \IfBooleanTF {#1}
    {
      \mkdn_section:nn {chapter}{#2}
    }
    {
      \mkdn_num_section:nn {chapter}{#2}
    }
  }
}
{
  \int_decr:N \l__mkdn_secoff_int
}

\NewDocumentCommand \section {s m}
{
  \IfBooleanTF {#1}
  {
    \mkdn_section:nn {section}{#2}
  }
  {
    \mkdn_num_section:nn {section}{#2}
  }
}

\NewDocumentCommand \subsection {s m}
{
  \IfBooleanTF {#1}
  {
    \mkdn_section:nn {subsection}{#2}
  }
  {
    \mkdn_num_section:nn {subsection}{#2}
  }
}

\NewDocumentCommand \subsubsection {s m}
{
  \IfBooleanTF {#1}
  {
    \mkdn_section:nn {subsubsection}{#2}
  }
  {
    \mkdn_num_section:nn {subsubsection}{#2}
  }
}

\NewDocumentCommand \paragraph {s m}
{
  \IfBooleanTF {#1}
  {
    \mkdn_section:nn {paragraph}{#2}
  }
  {
    \mkdn_num_section:nn {paragraph}{#2}
  }
}

\NewDocumentCommand \subparagraph {s m}
{
  \IfBooleanTF {#1}
  {
    \mkdn_section:nn {subparagraph}{#2}
  }
  {
    \mkdn_num_section:nn {subparagraph}{#2}
  }
}

\cs_new_nopar:Npn \mkdn_section:nn #1#2
{
  \par
  \mkdn_section_header:nn {#1}{#2}
  \par
}

\cs_new_nopar:Npn \mkdn_section_header:nn #1#2
{
  \prg_replicate:nn {\int_use:c {l__mkdn_ #1 _int} + \l__mkdn_secoff_int} {\#}
  \c_space_token #2 \c_space_token
  \prg_replicate:nn {\int_use:c {l__mkdn_ #1 _int} + \l__mkdn_secoff_int} {\#}
}

\cs_set_eq:NN  \mkdn_num_section:nn \mkdn_section:nn


% Indentation handling, we use a prefix to ensure that we get
% the right indentation
\tl_new:N \l__mkdn_indentpre_tl
\tl_new:N \l__mkdn_indentsep_tl
\tl_set:Nn \l__mkdn_indentpre_tl {X}
\tl_set:Nn \l__mkdn_indentsep_tl {Y}

\tl_new:N \l__mkdn_indent_tl

\cs_new:Npn \mkdn_add_indent:n #1
{
  \tl_put_right:Nn \l__mkdn_indent_tl {#1}
}

\cs_generate_variant:Nn \mkdn_add_indent:n {V}

\int_new:N \l__mkdn_indents_int
\int_set:Nn \l__mkdn_indents_int {0}

\cs_new:Npn \mkdn_push_indent:
{
  \tl_set_eq:cN {mkdn_indent_stack_ \int_use:N \l__mkdn_indents_int} \l__mkdn_indent_tl
  \int_gincr:N \l__mkdn_indents_int
}

\cs_new:Npn \mkdn_pop_indent:
{
  \int_gdecr:N \l__mkdn_indents_int
  \tl_set_eq:Nc \l__mkdn_indent_tl {mkdn_indent_stack_ \int_use:N \l__mkdn_indents_int}
}

\cs_new:Npn \mkdn_last_indent:
{
  \tl_set_eq:Nc \l__mkdn_indent_tl {mkdn_indent_stack_ \int_eval:n {\l__mkdn_indents_int - 1}}
}

\cs_new:Npn \mkdn_set_everypar:
{
  \everypar{
    \tl_use:N \l__mkdn_indent_tl\tl_use:N \l__mkdn_indentsep_tl
  }
}

\cs_new:Npn \mkdn_indent_by:n #1
{
  \tl_clear:N \l_tmpa_tl
  \prg_replicate:nn {#1} {\tl_put_right:NV \l_tmpa_tl  \l__mkdn_indentpre_tl}
  \mkdn_add_indent:V \l_tmpa_tl
}

\AtBeginDocument{\mkdn_set_everypar:}


% Not sure these are right
\NewDocumentEnvironment{quote}{}
{
  \par
  \mkdn_add_indent:n{>X}
}
{
  \par
}

\NewDocumentEnvironment{quotation}{}
{
  \par
  \mkdn_add_indent:n{>X}
}
{
  \par
}


% List environments
\cs_new_nopar:Npn \mkdn_save_lengths:
{
  \tl_set:Nx \l__mkdn_lengths_tl
  {
    \exp_not:N \parindent \parindent\relax
  }
}

\tl_new:N \l__mkdn_prelist_tl
\tl_new:N \l__mkdn_preitem_tl
\tl_new:N \l__mkdn_postitem_tl
\tl_new:N \l__mkdn_itempar_tl
\tl_set:Nn \l__mkdn_itempar_tl {\par\mbox{}}
\cs_new_nopar:Npn \mkdn_item_setlabel:n #1 {#1}

\NewDocumentEnvironment{markdown list}{m m}
{
  \tl_set:Nn \@itemlabel {#1}
  \tl_set:Nn \l__mkdn_itemprefix_tl {*~}
  \mkdn_push_indent:
  \mkdn_save_lengths:
  \tl_use:N \l__mkdn_prelist_tl
  #2\relax
  \tl_use:N \l__mkdn_lengths_tl
  \cs_set_eq:NN \@item \mkdn_item:w
}
{
  \mkdn_pop_indent:
  \par
}

\cs_new_nopar:Npn \mkdn_item:w [#1]
{
  \par
  \tl_use:N \l__mkdn_preitem_tl
  \tl_set_eq:NN \l_tmpa_tl \l__mkdn_indent_tl
  \mkdn_last_indent:
  \tl_use:N \l__mkdn_itempar_tl
  \tl_use:N \l__mkdn_itemprefix_tl
  \mkdn_item_setlabel:n {#1}
  \tl_set_eq:NN \l__mkdn_indent_tl \l_tmpa_tl
  \tl_use:N \l__mkdn_postitem_tl
}

\cs_set_eq:NN \mkdn_orig_list: \list
\cs_set_eq:NN \mkdn_orig_endlist: \endlist

\cs_set_eq:Nc \list {markdown list}
\cs_set_eq:Nc \endlist {end markdown list}

\RenewDocumentEnvironment{itemize} {}
{
  \use:c {markdown list}{}{\mkdn_indent_by:n {2}}
}
{
  \use:c {end markdown list}
}

\RenewDocumentEnvironment{enumerate} {}
{
  \use:c {markdown list}{}{
    \tl_set:Nn \l__mkdn_itemprefix_tl {1.~} \mkdn_indent_by:n {3}}
}
{
  \use:c {end markdown list}
}

\NewDocumentEnvironment{description} {}
{
  \cs_set:Npn \mkdn_item_setlabel:n ##1
  {
    \textbf{##1}
  }
  \use:c {markdown list}{}{\mkdn_indent_by:n {2}}
}
{
  \use:c {end markdown list}
}

% Entity handling
\NewDocumentCommand \entity {m}
{
  \&#1;
}

\NewDocumentCommand \hexentity {m}
{
  \entity{\#x#1}
}

% Are we using unicode?  Can we test for this?
\cs_set_eq:NN \mkdn_tilde: \~
\RenewDocumentCommand \textasciitilde {} {\mkdn_tilde: {} }

\RenewDocumentCommand \" {m} {\entity{#1uml}}
\RenewDocumentCommand \` {m} {\entity{#1grave}}
\RenewDocumentCommand \' {m} {\entity{#1acute}}
\RenewDocumentCommand \^ {m} {\entity{#1circ}}
\RenewDocumentCommand \~ {m} {\entity{#1tilde}}
\RenewDocumentCommand \c {m} {\entity{#1cedil}}
\RenewDocumentCommand \v {m} {\entity{#1caron}}
\RenewDocumentCommand \. {m} {\entity{#1dot}}
\RenewDocumentCommand \ae {} {\entity{aelig}}
\RenewDocumentCommand \AE {} {\entity{AElig}}
\RenewDocumentCommand \oe {} {\entity{oelig}}
\RenewDocumentCommand \OE {} {\entity{OElig}}
\RenewDocumentCommand \aa {} {\entity{aring}}
\RenewDocumentCommand \AA {} {\entity{Aring}}
\RenewDocumentCommand \o {} {\entity{oslash}}
\RenewDocumentCommand \O {} {\entity{Oslash}}
\RenewDocumentCommand \ss {} {\entity{szlig}}
\RenewDocumentCommand \P {} {\entity{\#182}}

\RenewDocumentCommand \TeX {} {TeX}
\RenewDocumentCommand \LaTeX {} {La\TeX}

% Should we set `\nobreakspace` directly?
\group_begin:
\char_set_catcode_active:n {126}
\AtBeginDocument{\cs_set:Npn ~ {\entity{nbsp}}}
\group_end:


% Hyperref options
\PassOptionsToPackage{customdriver=hmkdwn}{hyperref}
\PassOptionsToPackage{pageanchor=false}{hyperref}
\PassOptionsToPackage{raiselinks=false}{hyperref}

% Verbatim handling ...

\RenewDocumentCommand \verb {v}
{
  `#1`
}

\cs_set_eq:NN \mkdn_orig_verbatim: \verbatim
\cs_set_eq:NN \mkdn_orig_endverbatim: \endverbatim

\cs_set:Npn \mkdn_verbatim:
{
  \parskip\z@skip
  \@@par
  \@tempswafalse
  \def\par{
    \if@tempswa
      \leavevmode \null \@@par\penalty\interlinepenalty
    \else
      \@tempswatrue
      \ifhmode\@@par\penalty\interlinepenalty\fi
    \fi
  }
  \let\do\@makeother \dospecials
  \obeylines \verbatim@font \@noligs
  \hyphenchar\font\m@ne
  \everypar \expandafter{\the\everypar \unpenalty}
}

\txt_set_hook:nnn {mkdn} {verbatim} {\mkdn_indent_by:n {4}}

\cs_set:Npn \verbatim
{
  \vskip\parskip
  \group_begin:
  \txt_use_hook:nn {mkdn} {verbatim}
  \cs_set_eq:NN \@verbatim \mkdn_verbatim:
  \mkdn_orig_verbatim:
}

\cs_set:Npn \endverbatim
{
  \mkdn_orig_endverbatim:
  \txt_use_hook:nn {mkdn} {end verbatim}
  \group_end:
  \par
}


% Graphics handling (crude for now)

\tl_new:N \mkdn_img_prefix_tl

\NewDocumentCommand \mkdn_includegraphics {o m}
{
  ![](\tl_use:N \mkdn_img_prefix_tl #2.png)
}

\NewDocumentCommand \mkdnSetImagePrefix { m }
{
  \tl_set:Nn \mkdn_img_prefix_tl { #1 }
}

\AtBeginDocument{
  \cs_if_exist:NT \includegraphics
  {
    \cs_set_eq:NN \includegraphics \mkdn_includegraphics
  }
}

% disable centering

\RenewDocumentEnvironment {center} {}
{
  \par
}
{
  \par
}

\cs_set_eq:NN \centering \relax

\cs_set_eq:NN \maketitle \relax

\NewDocumentEnvironment {abstract} {}
{
  \section{Abstract}
}
{
}

\newcounter{figure}
\tl_set:Nn \ext@figure {lof}
\tl_set:Nn \fnum@figure{\figurename\nobreakspace\thefigure}
\tl_set:Nn \figurename {Figure}
\DeclareDocumentEnvironment {figure} { O{} }
{
  \def\@captype{figure}
}
{
}

\cs_new_nopar:Npn \@makecaption #1 #2
{
  \emph{#1:~ #2}\par
}

%\cs_new_nopar:Npn \mkdn_caption:n #1
%{
%}
%\AtBeginDocument{\cs_set_eq:NN \caption \mkdn_caption:n}


\endinput

% Local Variables:
% mode: latex3
% End: