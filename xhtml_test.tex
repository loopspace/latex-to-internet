\documentclass[
  text,
  xhtml,
  itex
]{internet}

\usepackage{lipsum}
\usepackage{hyperref}

\addcss+span.eqnnumber:before {content: "(";}+
\addcss+span.eqnnumber:after {content: ")";}+
\addcss+tr.eqnnumber {vertical-align: middle;}+
\addcss+table.displaymaths {width: 100%;}+
\addcss+table.displaymaths > tr {vertical-align: top;}+
\addcss+table.displaymaths td.label {width: 2em; vertical-align: middle;}+
\addcss+table.displaymaths td.label table {width: 100%; text-align:right;}+

\addjavascript@
function moveEqNums() {
  var dmaths = document.getElementsByClassName('displaymaths align');
  for (var i = 0; i < dmaths.length; i++) {
    var dmath = dmaths[i];
    if (dmath.tagName == 'table')
    {
      var rows = dmath.firstChild.childNodes[1].children[0].firstChild.firstChild.children;
      var labels = dmath.firstChild.childNodes[2].firstChild.children;
        for (var j = 0; j < rows.length; j++) {
          labels[j].style.height = rows[j].scrollHeight + 'px';
        }
    }
  }
}

window.onload = moveEqNums;
@


\newtheorem{theorem}{Theorem}

\renewcommand\footnote[1]{}

\begin{document}

\section{XHTML Test}

\subsection{Text}

A paragraph with \emph{emphasised} and \textbf{bold} text.
Some \verb+inline code+.
We can also handle footnotes\footnote{with aplomb}.
As we can use styles, we can handle \emph{upright \textup{text} inside emphasised text}.
\begin{center}
Thanks to attribute lists, we can handle centred text.
\end{center}

\subsection{Lists}

\subsubsection{Itemize}

\begin{itemize}
\item First item.

With a following paragraph\footnote{that has a footnote}.

\item Second item.

With another following paragraph.
\end{itemize}

\subsubsection{Enumerate}

\begin{enumerate}
\item First item.

With a following paragraph.

\item Second item.

With another following paragraph.
\end{enumerate}

\subsubsection{Description}

\begin{description}
\item[First] First item.

With a following paragraph.

\item[Second] Second item.

With another following paragraph.
\end{description}


\subsection{Verbatim}

\begin{verbatim}
A verbatim paragraph.
With @(#&$u&)!#<> symbols
\end{verbatim}

\begin{itemize}
\item Inside a list:

\begin{verbatim}
Some verbatim text.
\end{verbatim}

\end{itemize}

\subsection{Theorems}

\begin{theorem}
There are an infinitude of primes.
\end{theorem}

\begin{proof}
Ask Euclid.
\end{proof}

\subsection{Mathematics}

Some inline maths: \(\sin^2(x) + \cos^2(x) = 1\)

Some display maths:

\[
  \sin^2(x) + \cos^2(x) = 1
\]

and so on with more inline: \(\sum_1^\infty n\)

\begin{align*}
x &= y + z \\
a &= b + c
\end{align*}

Testing not: \(a \not\cong b\) \(a \not\in b\) \(a \not\le b\)

\begin{equation}
\label{eq:pyth}
a^2 + b^2 = c^2
\end{equation}

And let's refer back to \ref{eq:pyth}.
What about multiline numbered equations?

\begin{align}
\label{eq:one}
x &= y + z \\
\label{eq:two}
a &= \int b + c \\
\label{eq:three}
d &= e + f
\end{align}

And we can refer back to \ref{eq:one} and \ref{eq:two} and \ref{eq:three}.

\begin{align}
\label{eq:four}
x &= y + z \\
\nonumber
\label{eq:five}
a &= \int b + c \\
\label{eq:six}
d &= e + f
\end{align}

Equations \ref{eq:four}, \ref{eq:five}, and \ref{eq:six}.

\begin{equation}
\sum_{i = 0}^\infty a_i
\end{equation}

\subsection{Tabular}

\begin{tabular}{rrr}
A & table \\
with & entries \\ \hline
and & lines
\end{tabular}

\subsection{Filler}

\lipsum

\end{document}