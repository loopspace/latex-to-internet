\documentclass[
  text,
  xhtml,
  itex
]{internet}

\usepackage{hyperref}


\newtheorem{theorem}{Theorem}

\renewcommand\footnote[1]{}

\begin{document}

\section{XHTML Test}

\subsection{Text}

A paragraph with \emph{emphasised} and \textbf{bold} text.
Some \verb+inline code+.
We can also handle footnotes\footnote{with aplomb}.
As we can use styles, we can handle \emph{upright \textup{text} inside emphasised text}.
\begin{center}
Thanks to attribute lists, we can handle centred text.
\end{center}

\subsection{Lists}

\subsubsection{Itemize}

\begin{itemize}
\item First item.

With a following paragraph\footnote{that has a footnote}.

\item Second item.

With another following paragraph.
\end{itemize}

\subsubsection{Enumerate}

\begin{enumerate}
\item First item.

With a following paragraph.

\item Second item.

With another following paragraph.
\end{enumerate}

\subsubsection{Description}

\begin{description}
\item[First] First item.

With a following paragraph.

\item[Second] Second item.

With another following paragraph.
\end{description}


\subsection{Verbatim}

\begin{verbatim}
A verbatim paragraph.
With @(#&$u&)!#<> symbols
\end{verbatim}

\begin{itemize}
\item Inside a list:

\begin{verbatim}
Some verbatim text.
\end{verbatim}

\end{itemize}

\subsection{Theorems}

\begin{theorem}
There are an infinitude of primes.
\end{theorem}

\begin{proof}
Ask Euclid.
\end{proof}

\subsection{Mathematics}

Some inline maths: \(\sin^2(x) + \cos^2(x) = 1\)

Some display maths:

\[
  \sin^2(x) + \cos^2(x) = 1
\]

and so on with more inline: \(\sum_1^\infty n\)

\begin{align*}
x &= y + z \\
a &= b + c
\end{align*}

Testing not: \(a \not\cong b\) \(a \not\in b\) \(a \not\le b\)

\subsection{Tabular}

\begin{tabular}{rrr}
A & table \\
with & entries \\ \hline
and & lines
\end{tabular}

\end{document}