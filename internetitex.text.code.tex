% iTex, prefix: itex

\tl_new:N \l__itex_beginmaths_tl
\tl_new:N \l__itex_endmaths_tl
\tl_new:N \l__itex_begindisplay_tl
\tl_new:N \l__itex_enddisplay_tl
\tl_new:N \l__itex_begininline_tl
\tl_new:N \l__itex_endinline_tl

% Define the options for the itex module
\keys_define:nn {itex options}
{
  at begin maths .tl_set:N = \l__itex_beginmaths_tl,
  at end maths .tl_set:N = \l__itex_endmaths_tl,
  at begin display .tl_set:N = \l__itex_begindisplay_tl,
  at end display  .tl_set:N = \l__itex_enddisplay_tl,
  at begin inline .tl_set:N = \l__itex_begininline_tl,
  at end inline  .tl_set:N = \l__itex_endinline_tl,
}
% Get and set the options for the markdown-extra module
\prop_get:NnNT  \l__txt_popts_prop {itex} \l_tmpa_tl
{
  \keys_set:nV {itex options} \l_tmpa_tl
}

\char_set_catcode_active:N \$%$

\cs_new_nopar:Npn \itex_superscript:n #1
{
  \tl_use:N \l__txt_caret_tl \{#1\}
}

\cs_new_nopar:Npn \itex_subscript:n #1
{
  \tl_use:N \l__txt_underscore_tl \{#1\}
}

\cs_new_nopar:Npn \itex_percent:
{
  \@backslashchar \@percentchar
}

\group_begin:
\char_set_catcode_other:N \&
\cs_new_nopar:Npn \itex_amp: {&}
\group_end:


\bool_new:N \l__itex_display_bool
\bool_new:N \l__itex_maths_bool
\bool_new:N \l__itex_numbered_bool
\bool_new:N \l__itex_text_bool

\cs_new_nopar:Npn \itex_maths_start:
{
  \tl_use:N \l__itex_beginmaths_tl
  \bool_if:NTF \l__itex_display_bool
  {
    \tl_use:N \l__itex_begindisplay_tl
    \txt_newline:
    \$\$
    \txt_newline:
  }
  {
    \tl_use:N \l__itex_begininline_tl
    \$%$
  }
}

\cs_new_nopar:Npn \itex_maths_end:
{
  \bool_if:NTF \l__itex_display_bool
  {
    \txt_newline:
    \$\$
    \txt_newline:
    \tl_use:N \l__itex_enddisplay_tl
  }
  {
    \$%$
    \tl_use:N \l__itex_endinline_tl
  }
  \tl_use:N \l__itex_endmaths_tl
  \group_end:
}

\cs_new_nopar:Npn \itex_set_catcodes:
{
  \cs_set:Npx \itex_unset_catcodes:
  {
    \exp_not:N \char_set_catcode:nn {`\exp_not:N \^} {\char_value_catcode:n {`\^}}
    \exp_not:N \char_set_catcode:nn {`\exp_not:N \_} {\char_value_catcode:n {`\_}}
    \exp_not:N \char_set_catcode:nn {`\exp_not:N \>} {\char_value_catcode:n {`\>}}
    \exp_not:N \char_set_catcode:nn {`\exp_not:N \<} {\char_value_catcode:n {`\<}}
    \exp_not:N \char_set_catcode:nn {`\exp_not:N \&} {\char_value_catcode:n {`\_}}
  }

  \char_set_catcode_active:N \^
  \char_set_catcode_active:N \_
  \char_set_catcode_active:N \>
  \char_set_catcode_active:N \<
  \char_set_catcode_active:N \&
}

\NewDocumentCommand \itexSetCatcodes {}
{
  \itex_set_catcodes:
}

\NewDocumentCommand \itexRestoreCatcodes {}
{
  \itex_unset_catcodes:
}

\group_begin:
\char_set_catcode_active:N \^
\char_set_catcode_active:N \>
\char_set_catcode_active:N \<
\char_set_catcode_active:N \&
\char_set_catcode_active:N \;
\char_set_lccode:nn {`;}{`_}
\tl_to_lowercase:n {
\group_end:
  \cs_new_nopar:Npn \itex_set_actives:
  {
    \cs_set_eq:NN ^ \itex_superscript:n
    \cs_set_eq:NN ; \itex_subscript:n
    \cs_set_eq:NN > \gt
    \cs_set_eq:NN < \lt
    \cs_set_eq:NN & \itex_amp:
  }
}

\cs_new_nopar:Npn \itex_maths_init:
{
  \mathfont
  \itex_set_catcodes:
  \itex_set_actives:
  \cs_set_eq:NN \% \itex_percent:
  \cs_set_eq:NN \\ \itexnl
  \cs_set_eq:NN \align \itex_align:
  \bool_set_true:N \l__itex_maths_bool
}

\RenewDocumentCommand \( {} {${}}%$
\RenewDocumentCommand \) {} {${}}%$
\RenewDocumentCommand \[ {} {$$}%$
\RenewDocumentCommand \] {} {$$}%$

\cs_new_protected_nopar:Npn \itex_dollar:
{
  \bool_if:NTF \l__itex_maths_bool
  {
    \itex_maths_end:
  }
  {
    \group_begin:
    \itex_maths_init:
  }
  \peek_meaning:NTF $%$
  {
    \itex_double_dollar:w
  }
  {
    \itex_single_dollar:
  }
}

\cs_new:Npn \itex_double_dollar:w $%$
{
  \bool_if:NT \l__itex_maths_bool
  {
    \bool_set_true:N \l__itex_display_bool
    \itex_maths_start:
  }
}

\cs_new:Npn \itex_single_dollar:
{
  \bool_if:NT \l__itex_maths_bool
  {
    \itex_maths_start:
  }
}

\cs_set_eq:NN $ \itex_dollar: %$

\NewDocumentCommand \itexmathscommand {m o o}
{
  \cs_if_exist:NT #1
  {
    \cs_set_eq:cN {itex_text_\cs_to_str:N #1:} #1
  }
  \IfNoValueTF {#2}
  {
    \newtextcommand #1
  }
  {
    \IfNoValueTF {#3}
    {
      \newtextcommand #1 [#2]
    }
    {
      \newtextcommand #1 [#2] [{#3}]
    }
  }
  \cs_set_eq:cN {itex_maths_\cs_to_str:N #1:} #1
  \cs_set_nopar:Npn #1
  {
    \bool_if:NTF \itex_maths_bool
    {
      \use:c {itex_maths_\cs_to_str:N #1:}
    }
    {
      \use:c {itex_text_\cs_to_str:N #1:}
    }
  }
}

\cs_new_nopar:Npn \itex_label:n #1
{
  \txt_only_chars:nn {alpha} {#1}
  \string\label \{\tl_use:N \l__txt_chars_tl \}
}

\group_begin:
\cs_set_eq:NN \itex_orig_begin: \begin
\cs_set_eq:NN \itex_orig_end: \end
\newtextcommand \begin [1]
\newtextcommand \end
\cs_gset_eq:NN \itex_begin:n \begin
\cs_gset_eq:NN \itex_end:n \end
\cs_gset_eq:NN \begin \itex_orig_begin:
\cs_gset_eq:NN \end \itex_orig_end:
\group_end:

\NewDocumentEnvironment {itex equation} {}
{
  \bool_set_true:N \l__itex_numbered_bool
  \[
}
{
  \]
}

\cs_new_nopar:Npn \itex_restore_equation:
{
  \cs_set_eq:Nc \equation {itex equation}
  \cs_set_eq:Nc \endequation {end itex equation}
}

% hyperref redefines equation
\AtBeginDocument{
  \itex_restore_equation:
}

\DeclareDocumentCommand \ensuremath {}
{
  \itex_ensuremaths:
}

\cs_new_nopar:Npn \itex_ensuremaths:
{
  \bool_if:NTF \l__itex_maths_bool
  {
    \use:n
  }
  {
    $ \itex_ensuremaths_end:n %$
  }
}

\cs_new_nopar:Npn \itex_ensuremaths_end:n #1
{
  #1$%$
}

\NewDocumentEnvironment {align*} {}
{
  \bool_set_false:N \l__itex_numbered_bool
  $$
  \begin{aligned}
  \txt_newline:
}
{
  \txt_newline:
  \end{aligned}
  $$
}

\NewDocumentEnvironment {align} {}
{
  \bool_set_true:N \l__itex_numbered_bool
  \[
  \begin{aligned}
  \txt_newline:
}
{
  \txt_newline:
  \end{aligned}
  \]
}

\NewDocumentEnvironment {gather*} {}
{
  \bool_set_false:N \l__itex_numbered_bool
  $$
  \begin{gathered}
  \txt_newline:
}
{
  \txt_newline:
  \end{gathered}
  $$
}

\NewDocumentEnvironment {gather} {}
{
  \bool_set_true:N \l__itex_numbered_bool
  $$
  \begin{gathered}
  \txt_newline:
}
{
  \txt_newline:
  \end{gathered}
  $$
}

\NewDocumentEnvironment {multline*} {}
{
  \bool_set_false:N \l__itex_numbered_bool
  $$
  \begin{aligned}
  \txt_newline:
}
{
  \txt_newline:
  \end{aligned}
  $$
}

\NewDocumentEnvironment {multline} {}
{
  \bool_set_true:N \l__itex_numbered_bool
  \[
  \begin{aligned}
  \txt_newline:
}
{
  \txt_newline:
  \end{aligned}
  \]
}

\NewDocumentEnvironment {equation*} {}
{
  \bool_set_false:N \l__itex_numbered_bool
  $$
}
{
  $$
}

\RenewDocumentCommand \textrm {m}
{
  \text{#1}
}

\NewDocumentCommand \intertext {m}
{
  \\
  \text{#1}
  \\
}

\NewDocumentCommand \DeclareMathOperator {m m}
{
  \NewDocumentCommand #1 {} {\@backslashchar #2}
}

\NewDocumentCommand \DeclareiTeXOperator {m}
{
  \newtextcommand #1
}

% The `\not` command is quite complicated, this attempts to emulate it

\RenewDocumentCommand \not {}
{
  \peek_after:Nw \itex_if_not:
}

\cs_new:Npn \itex_if_not:
{
  \token_if_macro:NTF \l_peek_token
  {
    \itex_test_not:n
  }
  {
    \itex_orig_not:
  }
}

\cs_new:Npn \itex_test_not:n #1
{
  \cs_if_exist_use:cF {not \cs_to_str:N #1}
  {
    \cs_if_exist_use:cF {n \cs_to_str:N #1}
    {
      \itex_orig_not: #1
    }
  }
}

% Avoid nesting maths inside the `\text` command

\tl_new:N \l__itex_endtext_tl
\tl_set:Nn \l__itex_endtext_tl {\}\bool_set_false:N \l__itex_text_bool}

\DeclareDocumentCommand \text {}
{
  \peek_meaning_ignore_spaces:NTF \c_group_begin_token
  {
    \itex_text_group:
  }
  {
    \itex_text:n
  }
}

\cs_new_nopar:Npn \itex_text:n #1
{
  \itextext \itexgroup {#1}
}

\cs_new_nopar:Npn \itex_text_group:
{
  \bool_set_true:N \l__itex_text_bool
  \c_group_begin_token
  \cs_set_eq:NN $ \itex_text_dollar: %$
  \bool_set_false:N \l__itex_maths_bool
  \group_insert_after:N \l__itex_endtext_tl
  \itextext
  \{
    \peek_meaning_remove_ignore_spaces:NF \c_group_begin_token {}
}

\cs_new_nopar:Npn \itex_text_dollar:
{
  \bool_if:NTF \l__itex_maths_bool
  {
    \{
      \bool_set_false:N \l__itex_maths_bool
  }
    {
      \}
    \bool_set_true:N \l__itex_maths_bool
  }
}

% Here followeth the list of itex commands,
% modulo a few clashes
\itexmathscommand{\{}
\itexmathscommand{\}}
\itexmathscommand{\quad}
\itexmathscommand{\qquad}
\itexmathscommand{\thinspace}
\itexmathscommand{\dots}
\itexmathscommand{\phantom}[1]

\newtextcommand{\left}
\newtextcommand{\right}
\newtextcommand{\big}
\newtextcommand{\bigr}
\newtextcommand{\Big}
\newtextcommand{\Bigr}
\newtextcommand{\bigg}
\newtextcommand{\biggr}
\newtextcommand{\Bigg}
\newtextcommand{\Biggr}
\newtextcommand{\bigl}
\newtextcommand{\Bigl}
\newtextcommand{\biggl}
\newtextcommand{\Biggl}
\newtextcommand{\mathrlap}[1]
\newtextcommand{\mathllap}[1]
\newtextcommand{\mathclap}[1]
\newtextcommand{\rlap}
\newtextcommand{\llap}
\newtextcommand{\ulap}
\newtextcommand{\dlap}
\newtextcommand{\infty}
\newtextcommand{\infinity}
\newtextcommand{\lbrace}
\newtextcommand{\rbrace}
\newtextcommand{\vert}
\newtextcommand{\Vert}
\newtextcommand{\setminus}
\newtextcommand{\backslash}
\newtextcommand{\smallsetminus}
\newtextcommand{\sslash}
\newtextcommand{\lfloor}
\newtextcommand{\lceil}
\newtextcommand{\lang}
\newtextcommand{\langle}
\newtextcommand{\rfloor}
\newtextcommand{\rceil}
\newtextcommand{\rang}
\newtextcommand{\rangle}
\newtextcommand{\uparrow}
\newtextcommand{\downarrow}
\newtextcommand{\updownarrow}
\newtextcommand{\prime}
\newtextcommand{\alpha}
\newtextcommand{\beta}
\newtextcommand{\gamma}
\newtextcommand{\delta}
\newtextcommand{\zeta}
\newtextcommand{\eta}
\newtextcommand{\theta}
\newtextcommand{\iota}
\newtextcommand{\kappa}
\newtextcommand{\lambda}
\newtextcommand{\mu}
\newtextcommand{\nu}
\newtextcommand{\xi}
\newtextcommand{\pi}
\newtextcommand{\rho}
\newtextcommand{\sigma}
\newtextcommand{\tau}
\newtextcommand{\upsilon}
\newtextcommand{\chi}
\newtextcommand{\psi}
\newtextcommand{\omega}
\newtextcommand{\backepsilon}
\newtextcommand{\varkappa}
\newtextcommand{\varpi}
\newtextcommand{\varrho}
\newtextcommand{\varsigma}
\newtextcommand{\vartheta}
\newtextcommand{\varepsilon}
\newtextcommand{\phi}
\newtextcommand{\varphi}
\newtextcommand{\arccos}
\newtextcommand{\arcsin}
\newtextcommand{\arctan}
\newtextcommand{\arg}
\newtextcommand{\cos}
\newtextcommand{\cosh}
\newtextcommand{\cot}
\newtextcommand{\coth}
\newtextcommand{\csc}
\newtextcommand{\deg}
\newtextcommand{\dim}
\newtextcommand{\exp}
\newtextcommand{\hom}
\newtextcommand{\ker}
\newtextcommand{\lg}
\newtextcommand{\ln}
\newtextcommand{\log}
\newtextcommand{\sec}
\newtextcommand{\sin}
\newtextcommand{\sinh}
\newtextcommand{\tan}
\newtextcommand{\tanh}
\newtextcommand{\det}
\newtextcommand{\gcd}
\newtextcommand{\inf}
\newtextcommand{\lim}
\newtextcommand{\liminf}
\newtextcommand{\limsup}
\newtextcommand{\max}
\newtextcommand{\min}
\newtextcommand{\Pr}
\newtextcommand{\sup}
\newtextcommand{\omicron}
\newtextcommand{\epsilon}
\newtextcommand{\cdot}
\newtextcommand{\Alpha}
\newtextcommand{\Beta}
\newtextcommand{\Delta}
\newtextcommand{\Gamma}
\newtextcommand{\digamma}
\newtextcommand{\Lambda}
\newtextcommand{\Pi}
\newtextcommand{\Phi}
\newtextcommand{\Psi}
\newtextcommand{\Sigma}
\newtextcommand{\Theta}
\newtextcommand{\Xi}
\newtextcommand{\Zeta}
\newtextcommand{\Eta}
\newtextcommand{\Iota}
\newtextcommand{\Kappa}
\newtextcommand{\Mu}
\newtextcommand{\Nu}
\newtextcommand{\Rho}
\newtextcommand{\Tau}
\newtextcommand{\mho}
\newtextcommand{\Omega}
\newtextcommand{\Upsilon}
\newtextcommand{\Upsi}
\newtextcommand{\iff}
\newtextcommand{\Longleftrightarrow}
\newtextcommand{\Leftrightarrow}
\newtextcommand{\impliedby}
\newtextcommand{\Leftarrow}
\newtextcommand{\implies}
\newtextcommand{\Rightarrow}
\newtextcommand{\hookleftarrow}
\newtextcommand{\embedsin}
\newtextcommand{\hookrightarrow}
\newtextcommand{\longleftarrow}
\newtextcommand{\longrightarrow}
\newtextcommand{\leftarrow}
\newtextcommand{\to}
\newtextcommand{\rightarrow}
\newtextcommand{\leftrightarrow}
\newtextcommand{\mapsto}
\newtextcommand{\map}
\newtextcommand{\nearrow}
\newtextcommand{\nearr}
\newtextcommand{\nwarrow}
\newtextcommand{\nwarr}
\newtextcommand{\searrow}
\newtextcommand{\searr}
\newtextcommand{\swarrow}
\newtextcommand{\swarr}
\newtextcommand{\neArrow}
\newtextcommand{\neArr}
\newtextcommand{\nwArrow}
\newtextcommand{\nwArr}
\newtextcommand{\seArrow}
\newtextcommand{\seArr}
\newtextcommand{\swArrow}
\newtextcommand{\swArr}
\newtextcommand{\darr}
\newtextcommand{\Downarrow}
\newtextcommand{\uparr}
\newtextcommand{\Uparrow}
\newtextcommand{\downuparrow}
\newtextcommand{\duparr}
\newtextcommand{\updarr}
\newtextcommand{\Updownarrow}
\newtextcommand{\leftsquigarrow}
\newtextcommand{\rightsquigarrow}
\newtextcommand{\dashleftarrow}
\newtextcommand{\dashrightarrow}
\newtextcommand{\curvearrowbotright}
\newtextcommand{\righttoleftarrow}
\newtextcommand{\lefttorightarrow}
\newtextcommand{\leftrightsquigarrow}
\newtextcommand{\upuparrows}
\newtextcommand{\rightleftarrows}
\newtextcommand{\rightrightarrows}
\newtextcommand{\curvearrowleft}
\newtextcommand{\curvearrowright}
\newtextcommand{\downdownarrows}
\newtextcommand{\leftarrowtail}
\newtextcommand{\rightarrowtail}
\newtextcommand{\leftleftarrows}
\newtextcommand{\leftrightarrows}
\newtextcommand{\Lleftarrow}
\newtextcommand{\Rrightarrow}
\newtextcommand{\looparrowleft}
\newtextcommand{\looparrowright}
\newtextcommand{\Lsh}
\newtextcommand{\Rsh}
\newtextcommand{\circlearrowleft}
\newtextcommand{\circlearrowright}
\newtextcommand{\twoheadleftarrow}
\newtextcommand{\twoheadrightarrow}
\newtextcommand{\nLeftarrow}
\newtextcommand{\nleftarrow}
\newtextcommand{\nLeftrightarrow}
\newtextcommand{\nleftrightarrow}
\newtextcommand{\nRightarrow}
\newtextcommand{\nrightarrow}
\newtextcommand{\downharpoonleft}
\newtextcommand{\downharpoonright}
\newtextcommand{\leftrightharpoons}
\newtextcommand{\rightleftharpoons}
\newtextcommand{\upharpoonleft}
\newtextcommand{\upharpoonright}
\newtextcommand{\xrightarrow}[2][]
\newtextcommand{\xleftarrow}[2][]
\newtextcommand{\xleftrightarrow}[2][]
\newtextcommand{\xLeftarrow}[2][]
\newtextcommand{\xRightarrow}[2][]
\newtextcommand{\xLeftrightarrow}[2][]
\newtextcommand{\xleftrightharpoons}[2][]
\newtextcommand{\xrightleftharpoons}[2][]
\newtextcommand{\xhookleftarrow}[2][]
\newtextcommand{\xhookrightarrow}[2][]
\newtextcommand{\xmapsto}[2][]
\newtextcommand{\ldots}
\newtextcommand{\cdots}
\newtextcommand{\ddots}
\newtextcommand{\udots}
\newtextcommand{\vdots}
\newtextcommand{\colon}
\newtextcommand{\cup}
\newtextcommand{\union}
\newtextcommand{\bigcup}
\newtextcommand{\Union}
\newtextcommand{\cap}
\newtextcommand{\intersection}
\newtextcommand{\bigcap}
\newtextcommand{\Intersection}
\newtextcommand{\in}
\newtextcommand{\coloneqq}
\newtextcommand{\Coloneqq}
\newtextcommand{\coloneq}
\newtextcommand{\Coloneq}
\newtextcommand{\eqqcolon}
\newtextcommand{\Eqqcolon}
\newtextcommand{\eqcolon}
\newtextcommand{\Eqcolon}
\newtextcommand{\colonapprox}
\newtextcommand{\Colonapprox}
\newtextcommand{\colonsim}
\newtextcommand{\Colonsim}
\newtextcommand{\dblcolon}
\newtextcommand{\ast}
\newtextcommand{\Cap}
\newtextcommand{\Cup}
\newtextcommand{\circledast}
\newtextcommand{\circledcirc}
\newtextcommand{\curlyvee}
\newtextcommand{\curlywedge}
\newtextcommand{\divideontimes}
\newtextcommand{\dotplus}
\newtextcommand{\leftthreetimes}
\newtextcommand{\rightthreetimes}
\newtextcommand{\veebar}
\newtextcommand{\gt}
\newtextcommand{\lt}
\newtextcommand{\approxeq}
\newtextcommand{\backsim}
\newtextcommand{\backsimeq}
\newtextcommand{\barwedge}
\newtextcommand{\doublebarwedge}
\newtextcommand{\subset}
\newtextcommand{\subseteq}
\newtextcommand{\subseteqq}
\newtextcommand{\subsetneq}
\newtextcommand{\subsetneqq}
\newtextcommand{\varsubsetneq}
\newtextcommand{\varsubsetneqq}
\newtextcommand{\prec}
\newtextcommand{\parallel}
\newtextcommand{\nparallel}
\newtextcommand{\shortparallel}
\newtextcommand{\nshortparallel}
\newtextcommand{\perp}
\newtextcommand{\eqslantgtr}
\newtextcommand{\eqslantless}
\newtextcommand{\gg}
\newtextcommand{\ggg}
\newtextcommand{\geq}
\newtextcommand{\geqq}
\newtextcommand{\geqslant}
\newtextcommand{\gneq}
\newtextcommand{\gneqq}
\newtextcommand{\gnapprox}
\newtextcommand{\gnsim}
\newtextcommand{\gtrapprox}
\newtextcommand{\ge}
\newtextcommand{\le}
\newtextcommand{\leq}
\newtextcommand{\leqq}
\newtextcommand{\leqslant}
\newtextcommand{\lessapprox}
\newtextcommand{\lessdot}
\newtextcommand{\lesseqgtr}
\newtextcommand{\lesseqqgtr}
\newtextcommand{\lessgtr}
\newtextcommand{\lneq}
\newtextcommand{\lneqq}
\newtextcommand{\lnsim}
\newtextcommand{\lvertneqq}
\newtextcommand{\gtrsim}
\newtextcommand{\gtrdot}
\newtextcommand{\gtreqless}
\newtextcommand{\gtreqqless}
\newtextcommand{\gtrless}
\newtextcommand{\gvertneqq}
\newtextcommand{\lesssim}
\newtextcommand{\lnapprox}
\newtextcommand{\nsubset}
\newtextcommand{\nsubseteq}
\newtextcommand{\nsubseteqq}
\newtextcommand{\notin}
\newtextcommand{\ni}
\newtextcommand{\notni}
\newtextcommand{\nmid}
\newtextcommand{\nshortmid}
\newtextcommand{\preceq}
\newtextcommand{\npreceq}
\newtextcommand{\ll}
\newtextcommand{\ngeq}
\newtextcommand{\ngeqq}
\newtextcommand{\ngeqslant}
\newtextcommand{\nleq}
\newtextcommand{\nleqq}
\newtextcommand{\nleqslant}
\newtextcommand{\nless}
\newtextcommand{\supset}
\newtextcommand{\supseteq}
\newtextcommand{\supseteqq}
\newtextcommand{\supsetneq}
\newtextcommand{\supsetneqq}
\newtextcommand{\varsupsetneq}
\newtextcommand{\varsupsetneqq}
\newtextcommand{\approx}
\newtextcommand{\asymp}
\newtextcommand{\bowtie}
\newtextcommand{\dashv}
\newtextcommand{\Vdash}
\newtextcommand{\vDash}
\newtextcommand{\VDash}
\newtextcommand{\vdash}
\newtextcommand{\Vvdash}
\newtextcommand{\models}
\newtextcommand{\sim}
\newtextcommand{\simeq}
\newtextcommand{\nsim}
\newtextcommand{\smile}
\newtextcommand{\triangle}
\newtextcommand{\triangledown}
\newtextcommand{\triangleleft}
\newtextcommand{\cong}
\newtextcommand{\succ}
\newtextcommand{\nsucc}
\newtextcommand{\ngtr}
\newtextcommand{\nsupset}
\newtextcommand{\nsupseteq}
\newtextcommand{\propto}
\newtextcommand{\equiv}
\newtextcommand{\nequiv}
\newtextcommand{\frown}
\newtextcommand{\triangleright}
\newtextcommand{\ncong}
\newtextcommand{\succeq}
\newtextcommand{\succapprox}
\newtextcommand{\succnapprox}
\newtextcommand{\succcurlyeq}
\newtextcommand{\succsim}
\newtextcommand{\succnsim}
\newtextcommand{\nsucceq}
\newtextcommand{\nvDash}
\newtextcommand{\nvdash}
\newtextcommand{\nVDash}
\newtextcommand{\amalg}
\newtextcommand{\pm}
\newtextcommand{\mp}
\newtextcommand{\bigcirc}
\newtextcommand{\wr}
\newtextcommand{\odot}
\newtextcommand{\uplus}
\newtextcommand{\clubsuit}
\newtextcommand{\spadesuit}
\newtextcommand{\Diamond}
\newtextcommand{\diamond}
\newtextcommand{\sqcup}
\newtextcommand{\sqcap}
\newtextcommand{\sqsubset}
\newtextcommand{\sqsubseteq}
\newtextcommand{\sqsupset}
\newtextcommand{\sqsupseteq}
\newtextcommand{\Subset}
\newtextcommand{\Supset}
\newtextcommand{\ltimes}
\newtextcommand{\div}
\newtextcommand{\rtimes}
\newtextcommand{\bot}
\newtextcommand{\therefore}
\newtextcommand{\thickapprox}
\newtextcommand{\thicksim}
\newtextcommand{\varpropto}
\newtextcommand{\varnothing}
\newtextcommand{\flat}
\newtextcommand{\vee}
\newtextcommand{\because}
\newtextcommand{\between}
\newtextcommand{\Bumpeq}
\newtextcommand{\bumpeq}
\newtextcommand{\circeq}
\newtextcommand{\curlyeqprec}
\newtextcommand{\curlyeqsucc}
\newtextcommand{\doteq}
\newtextcommand{\doteqdot}
\newtextcommand{\eqcirc}
\newtextcommand{\fallingdotseq}
\newtextcommand{\multimap}
\newtextcommand{\pitchfork}
\newtextcommand{\precapprox}
\newtextcommand{\precnapprox}
\newtextcommand{\preccurlyeq}
\newtextcommand{\precsim}
\newtextcommand{\precnsim}
\newtextcommand{\risingdotseq}
\newtextcommand{\sharp}
\newtextcommand{\bullet}
\newtextcommand{\nexists}
\newtextcommand{\dagger}
\newtextcommand{\ddagger}
\newtextcommand[\not]{\itex_orig_not:}
\newtextcommand{\top}
\newtextcommand{\natural}
\newtextcommand{\angle}
\newtextcommand{\measuredangle}
\newtextcommand{\backprime}
\newtextcommand{\bigstar}
\newtextcommand{\blacklozenge}
\newtextcommand{\lozenge}
\newtextcommand{\blacksquare}
\newtextcommand{\blacktriangle}
\newtextcommand{\blacktriangleleft}
\newtextcommand{\blacktriangleright}
\newtextcommand{\blacktriangledown}
\newtextcommand{\ntriangleleft}
\newtextcommand{\ntriangleright}
\newtextcommand{\ntrianglelefteq}
\newtextcommand{\ntrianglerighteq}
\newtextcommand{\trianglelefteq}
\newtextcommand{\trianglerighteq}
\newtextcommand{\triangleq}
\newtextcommand{\vartriangleleft}
\newtextcommand{\vartriangleright}
\newtextcommand{\forall}
\newtextcommand{\bigtriangleup}
\newtextcommand{\bigtriangledown}
\newtextcommand{\nprec}
\newtextcommand{\aleph}
\newtextcommand{\beth}
\newtextcommand{\eth}
\newtextcommand{\ell}
\newtextcommand{\hbar}
\newtextcommand{\Im}
\newtextcommand{\imath}
\newtextcommand{\jmath}
\newtextcommand{\wp}
\newtextcommand{\Re}
\newtextcommand{\Perp}
\newtextcommand{\Vbar}
\newtextcommand{\boxdot}
\newtextcommand{\Box}
\newtextcommand{\square}
\newtextcommand{\emptyset}
%\newtextcommand{\empty} %%% conflicts with kvoptions.sty
\newtextcommand{\exists}
\newtextcommand{\circ}
\newtextcommand{\rhd}
\newtextcommand{\lhd}
\newtextcommand{\lll}
\newtextcommand{\unrhd}
\newtextcommand{\unlhd}
\newtextcommand{\Del}
\newtextcommand{\nabla}
\newtextcommand{\sphericalangle}
\newtextcommand{\heartsuit}
\newtextcommand{\diamondsuit}
\newtextcommand{\partial}
\newtextcommand{\qed}
\newtextcommand{\mod}
\newtextcommand{\pmod}
\newtextcommand{\bottom}
\newtextcommand{\neg}
\newtextcommand{\neq}
\newtextcommand{\ne}
\newtextcommand{\shortmid}
\newtextcommand{\mid}
\newtextcommand{\int}
\newtextcommand{\integral}
\newtextcommand{\iint}
\newtextcommand{\doubleintegral}
\newtextcommand{\iiint}
\newtextcommand{\tripleintegral}
\newtextcommand{\iiiint}
\newtextcommand{\quadrupleintegral}
\newtextcommand{\oint}
\newtextcommand{\conint}
\newtextcommand{\contourintegral}
\newtextcommand{\times}
\newtextcommand{\star}
\newtextcommand{\circleddash}
\newtextcommand{\odash}
\newtextcommand{\intercal}
\newtextcommand{\smallfrown}
\newtextcommand{\smallsmile}
\newtextcommand{\boxminus}
\newtextcommand{\minusb}
\newtextcommand{\boxplus}
\newtextcommand{\plusb}
\newtextcommand{\boxtimes}
\newtextcommand{\timesb}
\newtextcommand{\sum}
\newtextcommand{\prod}
\newtextcommand{\product}
\newtextcommand{\coprod}
\newtextcommand{\coproduct}
\newtextcommand{\otimes}
\newtextcommand{\Otimes}
\newtextcommand{\bigotimes}
\newtextcommand{\ominus}
\newtextcommand{\oslash}
\newtextcommand{\oplus}
\newtextcommand{\Oplus}
\newtextcommand{\bigoplus}
\newtextcommand{\bigodot}
\newtextcommand{\bigsqcup}
\newtextcommand{\biguplus}
\newtextcommand{\wedge}
\newtextcommand{\Wedge}
\newtextcommand{\bigwedge}
\newtextcommand{\Vee}
\newtextcommand{\bigvee}
\newtextcommand{\invamp}
\newtextcommand{\parr}
\newtextcommand{\frac}[2]
\newtextcommand{\tfrac}[2]
\newtextcommand{\binom}[2]
\newtextcommand{\tbinom}[2]
\newtextcommand{\tensor}
\newtextcommand{\multiscripts}
\newtextcommand{\overbrace}[1]
\newtextcommand{\underbrace}[1]
\newtextcommand{\underline}[1]
\newtextcommand{\bar}[1]
\newtextcommand{\overline}[1]
\newtextcommand{\closure}[1]
\newtextcommand{\widebar}[1]
\newtextcommand{\vec}[1]
\newtextcommand{\widevec}[1]
\newtextcommand{\dot}[1]
\newtextcommand{\ddot}[1]
\newtextcommand{\dddot}[1]
\newtextcommand{\ddddot}[1]
\newtextcommand{\tilde}[1]
\newtextcommand{\widetilde}[1]
\newtextcommand{\check}[1]
\newtextcommand{\widecheck}[1]
\newtextcommand{\hat}[1]
\newtextcommand{\widehat}[1]
\newtextcommand{\underset}[2]
\newtextcommand{\stackrel}[2]
\newtextcommand{\overset}[2]
\newtextcommand{\over}
\newtextcommand{\atop}
\newtextcommand{\underoverset}[3]
\newtextcommand{\sqrt}[1]
\newtextcommand{\root}
%\newtextcommand{\space}
\newtextcommand{\statusline}
\newtextcommand{\tooltip}
\newtextcommand{\toggle}
\newtextcommand{\fghilight}
\newtextcommand{\fghighlight}
\newtextcommand{\bghilight}
\newtextcommand{\bghighlight}
\newtextcommand{\color}
\newtextcommand{\bgcolor}
\newtextcommand{\displaystyle}
\newtextcommand{\textstyle}
\newtextcommand{\textsize}
\newtextcommand{\scriptsize}
\let\scriptstyle=\scriptsize
\newtextcommand{\scriptscriptsize}
\let\scriptscriptstyle=\scriptscriptsize
\newtextcommand{\mathit}[1]
\newtextcommand{\boldsymbol}
\newtextcommand{\mathbf}[1]
\newtextcommand{\mathrm}[1]
\newtextcommand{\mathbb}[1]
\newtextcommand{\mathfrak}[1]
\newtextcommand{\mathfr}[1]
\newtextcommand{\slash}[1]
\newtextcommand{\boxed}[1]
\newtextcommand{\mathcal}[1]
\newtextcommand{\substack}[1]
\newtextcommand{\arrayopts}[1]
\newtextcommand{\colalign}
\newtextcommand{\collayout}
\newtextcommand{\rowalign}
\newtextcommand{\equalrows}
\newtextcommand{\equalcols}
\newtextcommand{\rowlines}
\newtextcommand{\collines}
\newtextcommand{\frame}
\newtextcommand{\padding}
\newtextcommand{\rowopts}
\newtextcommand{\cellopts}
\newtextcommand{\rowspan}
\newtextcommand{\colspan}
\newtextcommand{\medspace}
\newtextcommand{\thickspace}
\newtextcommand{\negspace}
%\newtextcommand{\href}
\newtextcommand{\operatorname}
\newtextcommand{\mathop}
\newtextcommand{\mathbin}
\newtextcommand{\mathrel}
\newtextcommand{\includegraphics}
\newtextcommand[\text]{\itextext}
\newtextcommand[\array]{\itexarray}
\newtextcommand[\align]{\itexalign}
\newtextcommand[\space]{\itexspace}[3]
\newtextcommand[]{\itexgroup}[1]
\newtextcommand[\text]{\textrm}[1]
\newtextcommand[\\]{\itexnl}

\newtextenvironment{array}[1]
\newtextenvironment{cases}
\newtextenvironment{aligned}
\newtextenvironment{gathered}
\newtextenvironment{matrix}
\newtextenvironment{pmatrix}
\newtextenvironment{bmatrix}
\newtextenvironment{Bmatrix}
\newtextenvironment{vmatrix}
\newtextenvironment{Vmatrix}
\newtextenvironment{smallmatrix}
\newtextenvironment{svg}

\newtextcommand*{\!}
\newtextcommand*{\|}
\newtextcommand*{\:}
\newtextcommand*{\;}
\itexmathscommand{\,}

\cs_set_eq:NN \dotsc \dots
\cs_set_eq:NN \dotsb \dots
\cs_set_eq:NN \land \wedge
\cs_set_eq:NN \lor \vee

\endinput

% Local Variables:
% mode: latex3
% End: