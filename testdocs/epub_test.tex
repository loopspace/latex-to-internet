\documentclass[text,itex,epub]{internet}
%\documentclass[doc]{internet}

\usepackage{hyperref}

\title{An ePub}

\begin{document}

\tableofcontents

\section*{Introduction}

My aim here is to derive the bandwidth $\Delta\lambda$ of a Rugate notch filter on-axis and further show that the expressions for reflectance and bandwidth are identical to those of single mode Fiber Bragg Gratings. 

We start with the work of Southwell (1987).  Following him we define a Rugate filter as a  sinusoidal refractive index variation along coordinate $z$:
\[ n(z) = n_a + \frac{n_p}{2} \sin(K_1 z + \phi)  \] 
where $n_a$ is the average refractive index and $n_p$ is the peak to peak variation.
Southwell then does a  detailed coupled-wave analysis and derives 
the following expression for the reflectance profile of a Rugate filter:
\[
R= {  \kappa^2 \sinh^2(sL) \over
s^2 \cosh^2(sL) + (\alpha/2)^2 \sinh^2(sL)
}
\]
where for normal incidence ($\theta=0$) the coupling constant $\kappa$ is:
\[\kappa= \pm \frac{\pi}{2} \frac{n_p}{\lambda}  \]
(the $\pm$ corresponding to $s-$ and $p-$polarizations). The following expressions let us relate $s$ (which can be real or imaginary) and $\alpha$ 
to the wavelength of the light and the Rugate filter properties:
\[ s=\sqrt{\kappa^2 - (\alpha/2)^2} \]
\[ \alpha = 2 k_z - K_1 \]
\[k_z = \frac{2 \pi}{\lambda} n_a \]
\[ K_1  =  \frac{4 \pi}{\lambda_1} n_a = \frac{2 \pi}{\Lambda}\]


\end{document}
