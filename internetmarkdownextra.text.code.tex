% Markdown-Extra, prefix: mkex

% Built on top of markdown
\txt_load_file:n {markdown}

% Define the options for the markdown-extra module
\keys_define:nn {markdown extra options}
{
  unknown .code:n =
  {
    \txt_try_options:nnn {markdown} {\l_keys_key_tl} {#1}
  }
}

% Get and set the options for the markdown-extra module
\prop_get:NnNT  \l__txt_popts_prop {markdownextra} \l_tmpa_tl
{
  \keys_set:nV {markdown extra options} \l_tmpa_tl
}

% Better verbatim handling

\group_begin:
\char_set_lccode:nn {`;}{126}
\tl_to_lowercase:n {
\group_end:
\tl_set:Nn \l__mkex_ttilde_tl {;;;}
}

\txt_set_hook:nnn {mkdn} {verbatim} {\tl_use:N \l__mkex_ttilde_tl}
\txt_set_hook:nnn {mkdn} {end verbatim} {\tl_use:N \l__mkex_ttilde_tl}

% Native description lists

\RenewDocumentEnvironment{description} {}
{
  \cs_set:Npn \mkdn_item_setlabel:n ##1
  {
    ##1\txt_newline:
    :~~~
  }
  \use:c {markdown list}{}{
    \tl_set:Nn \l__mkdn_itemprefix_tl {} \mkdn_indent_by:n {4}}
}
{
  \use:c {end markdown list}
}

\int_new:N \l__mkex_footnotes_int
\tl_new:N \l__mkex_footnotes_tl

\tl_put_left:Nn \@enddocumenthook {\tl_use:N \l__mkex_footnotes_tl}

\RenewDocumentCommand \footnote {m}
{
  \int_gincr:N \l__mkex_footnotes_int
  [\l__txt_caret_tl \int_use:N \l__mkex_footnotes_int]
  \tl_gput_right:Nn \l__mkex_footnotes_tl
  {
    \par
    [\l__txt_caret_tl
    }
    \tl_gput_right:NV \l__mkex_footnotes_tl \l__mkex_footnotes_int
    \tl_gput_right:Nn \l__mkex_footnotes_tl {
    ]:~ #1
  }
}


\endinput

% Local Variables:
% mode: latex3
% End: